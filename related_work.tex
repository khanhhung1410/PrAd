
\section{Related Work}
\label{sec:related}


\textbf{Data Transformation}. In this technique, data is encoded before uploading to the LBS, who is able to perform search and query processing on encoded data. Clients, having access to secret encoding key, issue encoded queries to LBS. Since both records in database and queries are encoded and unreadable to LBS, location privacy is obtained.

\cite{Space_Transformation} utilized a Hilbert-based transformation to preprocess spatial data before storing it in the database. The transformation is considered as an encryption functions and its parameters serve the same role as encryption key. This technique can evaluate approximate Nearest Neighbor (NN) queries directly in transformed data points. Other work proposed by Wong et. al. \cite{secure_NN} uses a different transformation which preserves relative distances of all point POIs in the spatial database. This approach enables the LBS to answer accurate kNN queries. \cite{circular_shift_kNN} uses Moore curve and Paillier cryptosystem to perform a secret circular shifts of spatial data. The technique provides almost accurate answer for kNN queries. 

Because of encryption-like properties, data transformation approaches can reasonably protect users' location privacy. However, they are still vulnerable to access pattern attacks as the same encoded results always are always rendered for the same queries. Specifically, if the LBS can observe that a particular ciphertext is returned many times, and it also has external geographical knowledge of popular POIs in the area, it can make inference on true locations of ciphertext and invert the transformation.

\textbf{PIR-based Location Privacy}. The PIR concept was originally introduced by Chor et al. \cite{PIR_Origin} and has been extensively studied over years \cite{PIR2007, PIR2010, PIR2011, PIR_SC_2006}. This class of protocols nullifies access pattern attacks since the server doesn't know which item is requested.As discussed in subsection \ref{backgroundPIR}, there are two broad categories of PIR protocols, \textit{Cryptographic PIR} and \textit{Hardware-based PIR}. Cryptographic PIR categories can be further divided into two groups, which are \textit{information theoretic PIR} and \textit{computational PIR}. The first group is secure even against computationally unbounded adversary with an assumption that there is not collusion among servers. Computational PIR approaches are built based on computational intractability of well-known problems. Even though this class is more practical in the former, it still entails a prohibitive cost of processing. 

Secure hardware PIR is the most practical mechanism among all PIR techniques. \cite{Spiral} adopts this concept to protect location privacy in kNN queries. The main idea of this technique is to reduce a query processing to a set of PIR block retrieval executed by the trusted secure coprocessor. Though each block retrieval is completely private, the unstrusted LBS is still able to infer user's location by observing a number of PIR request for each query. This is because that different queries will have different cardinality of private retrievals. As the result, this technique fails to provide strong location privacy. Ghinita et. al. \cite{No_Need_Anonymizer} present a technique that can completely protect users' location privacy as each query incurs exactly one PIR request. However, this technique is limited to single NN queries and requires an excessive processing cost.


\textbf{Private Advertising System} Online and mobile advertising are very promising advertising markets. However, there are many privacy concerns in adopting these advertising models. There are several works on designing privacy-aware advertising system to protect users' privacy. The first class of these systems target personalized online advertisement on ordinary browsers. Adnostic \cite{adnostic} and Privad \cite{Privad} enable private advertising by maintaining users' profiles locally on their computers. The selection of ads shown on users' displays are performed based on these profiles. The billing procedure (ads view/click report) is performed using cryptographic techniques to prevent other parties to learn which ads are displayed to users. Juels \cite{Juels} presented another private advertising system focusing on the private distribution of ads. This scheme utilize PIR and mix network to protect users' privacy. Another private advertising system, call RePriv \cite{REPRIV}, is presented by Microsoft. This system performs personalization by mining browsing behaviour in privacy-conscious manner and provides users with explicit and precise control over the release of private information from their browsers.

Another class of private advertising system focuses on mobile advertising. MoRePriv \cite{MoRePriv} advocates for OS-level service to solve a conflict of privacy and content personalization on mobile devices. By combining personal preference mining techniques with a privacy filer, MoRePriv can provide necessary information for targeting advertising without compromise user privacy. Nath et. al. presents SmartAds \cite{SmartAds}, a contextual advertising system on mobile devices. SmartAds guarantees that the only thing the ad server knows is the ad keyword, which is considered as non-sensitive information while all other sensitive information is keep private. Other system, called MobiAd \cite{MobiAd}, is designed to build users' interest profiles on their phones, download and display relevant ads and reports clicks via DTN protocol. MobiAd maintains users' privacy by keeping user profile on the handset and routing ad reports around many intermediate node to make them anonymous. \cite{PAPMA1, PAPMA2} introduce flexible framework for personalized advertising in which users can limit the amount of private information they are willing to share with ad server.
Despite providing some level of privacy, these schemes still leak some user information to the server. Moreover, they pay very little attention to location privacy issue. Hence, none of these works can be directly applied to LBA without revealing users' locations.


\section{Introduction}
\label{sec:intro}

% Motivate why the problem is important and brieftly discuss what has been done (this is not supposed to be a "related work" part. just give a brief overview on other approaches)

A tremendous growth in smartphone usage has been witnessed in recent years and it is expected that more than one-third of global population will use smartphones within the next 3 years \cite{smartphone_growth}. In such a context, mobile-advertising has become a promising market. One of the most popular forms of mobile-advertising is \textit{location-based advertising} (LBA). This is a form of advertising that leverages location-based services to conduct mobile advertising. LBA offers a mechanism in which location-specific advertisements (those that are particularly relevant to a specific location) are delivered to appropriate consumers (those that are close to the advertisement's location, for example). For brevity, we simply refer to location-specific advertisements as \textit{ads}. 
Most of today's LBA services track users' personal and private information in order to be able to serve the most relevant ads to the users. Such tracking has raised many concerns about privacy violation. To a certain extent, \textit{location-based advertising server} (LBAS) has to take into consideration at least the locations of users and ads. However, LBAS should not be trusted as it may reveal users' locations to a third party without users' consent. Location disclosing has great implications in term of privacy \cite{No_Need_Anonymizer, DP_Spaial_Decomposition}. Given a location information of individuals, a broad set of other sensitive information such as health status or religious view could be inferred\cite{No_Need_Anonymizer}. These concerns raise a need of protecting location privacy in location-based advertising.

In this paper, we focus on a specific type of LBA that displays an ads on user's phone during their usage of ad-sponsored applications when they are in the ads' proximity. We target the highest level of location privacy, which completely protects users' location privacy from any unstrusted parties. The most prominent and powerful candidate of these unstrusted parties is LBAS since it has access to users' location information.

Various techniques have been proposed to protect location privacy in the context of location based services (LBS) \cite{k-anonymity1, k-anonymity2, New_Casper, Space_Transformation, Lee:2007, Yiu:2010, Spiral, PIR-kNN}. The main idea of these techniques is to enable location-based queries such as nearest neighbor queries or range query in such a way that actual locations of query points are not revealed. Among these solutions, there are three main categories. The first category \cite{k-anonymity1, k-anonymity2, New_Casper} employs \textit{location obfuscation} idea, in which a query point is either cloaked in a group of some other query points or blurred in an area so that the exact location of the query point cannot be inferred. However, in this class of techniques, user's location can be restricted in a small area of the space, thus it is relative easy to infer her location. The other category \cite{Space_Transformation, Lee:2007, Yiu:2010} uses \textit{space encoding} techniques to hide actual location of Point of Interests (POI). Approaches based on this concept are still not able to protect users' location information from inference attacks where the unstrusted LBAS may match users' queries with outliers or populate locations based on the access frequencies/patterns. The third major concept employs \textit{private information retrieval} (PIR) \cite{Spiral, No_Need_Anonymizer, PIR-kNN}. Generally, PIR-based approaches utilize PIR protocol \cite{PIR_Origin} to implement a query procedure in which database item is retrieved privately from location-based service without it learning which block was retrieved. Though this technique is resistant to attacks that the two other classes are vulnerable to, it has its own limitations. 
Approach proposed in \cite{Spiral} leaks the cardinality of the PIR request while scheme presented in \cite{No_Need_Anonymizer} incurs a prohibitive computational cost. On a different perspective, there are many studies on privacy issues in mobile-advertising \cite{PAPMA1, PAPMA2, MoRePriv}. However, these studies focus on content personalization and targeted advertising instead of location-based advertising. Thus, they do not try to protect location privacy of users.

%We argue that merely applying these techniques  is not sufficient to protect user's location privacy in context of location-based advertising as LBAS can infer user's location based on a set of ads retrieved, observing access frequencies of ads with respect to certain locations (inference attack) and history trajectories of user issuing queries (correlation attack). 

In this work, by adopting \textit{space encoding} and PIR techniques, we propose \codename, a novel LBA model targeting location privacy; i.e. enable location-based advertising without compromising location privacy of users. 
Our key insight is that instead of sending location information to LBAS and let it select appropriate ads to deliver to users, \codename keeps the sensitive information on user's phone, carry out the selection locally, and then privately request pertinent ads from LBAS. LBAS no longer obtains location information of users or processes spatial queries, which means it is deprived from sensitive information. Moreover, we design \codename such that ads retrieval and delivery are carried out privately, i.e. without LBAS knowing which are requested and retrieved. We introduce three main privacy metrics and justify that \codename satisfies all those three metrics, and thus can put the claim that \codename can obtain location privacy in LBA.

\codename encodes user's location using one-way space encoding function to get a location index that represents the location. The only data sent out of the device is this encoded index; i.e. no location information ever leaves the device. The space encoding technique is designed so that locality and neighborhood of spatial objects are preserved. Given that location information of ads are also encoded by the same techniques, it is because of this very property that ads selection could be performed locally on user's devices instead of being processed by LBAS as in traditional model. The mobile application then requests ads with relevant location index from the LBAS. Utilizing this technique alone can protect users' location privacy in snapshots. That is, given a single ads retrieval, LBAS cannot find out from where the request is made, i.e. where the user currently is. However, based on access frequencies of records in its database, history queries and external geographic knowledge, LBAS can carry out inference or correlation attacks to deduce user's location. We further improve \codename by adopting an ORAM-based \textit{private information retrieval} technique \cite{PIR_SC_2006} to completely nullifies inference and correlation attacks. That is, \codename leaves LBAS no opportunity to infer user's location. As the result, users are served with relevant ads while their locations are protected. Finally, \codename uses homomorphic encryption techniques to ensure proper and accurate accounting as well as billing among LBAS, application publisher and advertisers. The only assumption \codename makes is the presence of a piece of trusted-hardware called \textit{Secure Coprocessor} (SC) installed on LBAS's system.

In summary, we claim the following contributions in this work:

\begin{itemize}
\item We propose \codename, a framework enabling location-privacy aware LBA based on a set of state-of-the-art privacy preserving techniques.
\item We justify how \codename obtains the highest level of location-privacy in the context of LBA.
\item We evaluate the performance of \codename using real world dataset to show the scalability and practicality of our proposed framework.
\end{itemize}


% Private information retrieval using trusted hardware is refered to as a ORAM-based PIR mechanism in Building Castles out of Mud: Practical Access Pattern Privacy and Correctness on Untrusted Storage
%refer to hardward-based kNN in related work for introduction